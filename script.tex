% generated by Docutils <http://docutils.sourceforge.net/>
\documentclass[a4paper,english]{article}
\usepackage{fixltx2e} % LaTeX patches, \textsubscript
\usepackage{cmap} % fix search and cut-and-paste in PDF
\usepackage[T1]{fontenc}
\usepackage[utf8]{inputenc}
\usepackage{ifthen}
\usepackage{babel}

%%% Custom LaTeX preamble
% PDF Standard Fonts
\usepackage{mathptmx} % Times
\usepackage[scaled=.90]{helvet}
\usepackage{courier}

%%% User specified packages and stylesheets

%%% Fallback definitions for Docutils-specific commands

% hyperlinks:
\ifthenelse{\isundefined{\hypersetup}}{
  \usepackage[unicode,colorlinks=true,linkcolor=blue,urlcolor=blue]{hyperref}
  \urlstyle{same} % normal text font (alternatives: tt, rm, sf)
}{}
\hypersetup{
  pdftitle={Script},
}

%%% Body
\begin{document}

% Document title
\title{Script%
  \phantomsection%
  \label{script}}
\author{}
\date{}
\maketitle

% ----------

% Objectives

% ----------

% At the end of this video, you should know --

% 1. Discover about FOSSEE

% Prerequisites

% -------------

% None

% Authors              : Primal Pappachan and Satyajit Sarangi
% Internal Reviewer   : Sushma
% External Reviewer   :
% Checklist OK?       : <put date stamp here, if OK> [2011-09-26]

% L1

\{\{\{Show 'FOSSEE logo' slide\}\}\}

% R1

Hello, Welcome to FOSSEE - Free and Open Source Software for Science and Engineering Education. The goal of the project is to enable the students and faculty of science and engineering colleges alike, all across India to use open source software tools for all their computational needs, thus improving the quality of instruction and learning. We are based at IIT Bombay. It is funded by MHRD as part of the National Mission on Education through ICT with the thrust area being ``Adaptation and deployment of open source simulation packages equivalent to MATLAB, ORCAD etc''.

% L2

\{\{\{Show 'PI's' Slides\}\}\}

% R2

The Principal Investigators of the project are
\newcounter{listcnt0}
\begin{list}{\arabic{listcnt0}.}
{
\usecounter{listcnt0}
\setlength{\rightmargin}{\leftmargin}
}

\item Prof. Prabhu Ramachandran

\item Prof. Madhu Belur

\item Prof. Kannan Moudgalya

\item Prof. Mani Bhushan
\end{list}

% L3

\{\{\{Show Slides on 'What do we do?'\}\}\}

% R3

We work on the following deliverables in the FOSSEE project.
\setcounter{listcnt0}{0}
\begin{list}{\arabic{listcnt0}.}
{
\usecounter{listcnt0}
\setlength{\rightmargin}{\leftmargin}
}

\item SEES

\item Workshop for Coordinators on Software Development Techniques for Teachers of Engineering and Science Institues

\item Textbook Conversion Program

\item Spoken Tutorials
\end{list}

% L4

\{\{\{Show the 'SEES' slide\}\}\}

% R4

Engineering \& science students use computers for a large number of curricular tasks -{}- mostly computation centred. However, they do not see this as coding or programming tasks and usually are not even aware of the tools and techniques that will help them to handle these tasks better. This results in less than optimal use of their time and resources. This also causes difficulties when it comes to collaboration and building on other people's work.

This program is intended to train such students in good software practices and tools for producing code and documentation. SEES stands for Software Engineering for Engineers and Scientists.

The modules covered in this course are
\setcounter{listcnt0}{0}
\begin{list}{\arabic{listcnt0}.}
{
\usecounter{listcnt0}
\setlength{\rightmargin}{\leftmargin}
}

\item Using Linux Tools

\item Basic Python Programming

\item LaTeX

\item Version Control

\item Test Driven Development

\item Advanced Python
\end{list}

% L5

\{\{\{Show the 'Workshop' slide\}\}\}

% R5

This workshop is conducted for coordinators on ``Software development Techniques for Teachers of Engineering and Science Institutes''.  We invite expert faculty from various remote centers to a Coordinators training workshop which is held in IIT. These Coordinators then act as Workshop Coordinators during the main workshop, liaising between the participants at their Remote Centers and IIT Bombay, from where the workshop is transmitted live. The lecture transmission and live interaction take place through distance mode using the AVIEW technology, at selected remote centers across the country. During the main workshop, the Workshop Coordinator at every center supervises the conduct of tutorials and Labs. All the lectures and tutorial sessions are video recorded. The final edited audio-visual contents, along with other course material will be released under Open Source. These contents can be freely used later by all teachers and students.

% L6

\{\{\{Show 'Textbook conversion program' slide\}\}\}

% R6

The aim of this project is to create a repository of reference material in the form of solved problems for Scientific Computing with Open Source tools.

The initial plan is to take up a set of text books and have the (solved and unsolved) problems implemented/coded in an Open Source language like Python, Scilab or Sage.

This is how the project works. We have a bunch of textbooks, that we are targeting to code, in this iteration. Each book has a coordinator, who is in-charge of having the book coded in an Open Source language. He finds one or more students, who are willing to implement parts of the book and complete the book, roughly in the duration of one semester. The coordinator also finds a TA/reviewer, who will review the work done by the students and approve their work. The student then codes the problems from the book in the language that he chooses. The TA periodically reviews this work and helps him complete the task. The TA and student meet up in person, if and when required. Once the book is completed, the TA and the coordinator together, announce that the book is complete. This will then be reviewed by a member from FOSSEE and confirmed. Once, this work is announced as complete, all the participants get their goodies!

% L7

\{\{\{Show the 'Spoken Tutorials' slide\}\}\}

% R7

These are video tutorials which can be used for teaching or self learning purposes by teachers and students respectively. Presently, we have 37 video tutorials on scientific computing using python. We also aim to complete video tutorials for the complete SEES course. The videos can be downloaded from here.

Thank you for watching the video and for more details visit www.fossee.in.
.. L7

\{\{\{Show the 'Thank you' slide\}\}\}

Show the website ``www.fossee.in'' in the browser.

\end{document}
