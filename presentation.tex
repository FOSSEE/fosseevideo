% Beamer Presentation Template
% Author: Primal Pappachan
% Last Updated: 4 - 10 - 2011
% 

\documentclass[compress,red]{beamer} %

\usetheme{Warsaw}
% other themes: AnnArbor, Antibes, Bergen, Berkeley, Berlin, Boadilla, boxes, CambridgeUS, Copenhagen, Darmstadt, default, Dresden, Frankfurt, Goettingen,
% Hannover, Ilmenau, JuanLesPins, Luebeck, Madrid, Maloe, Marburg, Montpellier, PaloAlto, Pittsburg, Rochester, Singapore, Szeged, classic

\usepackage[latin1]{inputenc}
\usefonttheme{professionalfonts}
\usepackage{times}
\usepackage{tikz}
\usepackage{amsmath}
\usepackage{verbatim}
\usepackage{url}
\usepackage{graphics}

%\usecolortheme{default}
% color themes: albatross, beaver, beetle, crane, default, dolphin, dov, fly, lily, orchid, rose, seagull, seahorse, sidebartab, structure, whale, wolverine

\useoutertheme[subsection=false]{smoothbars}

\usefonttheme[onlysmall]{structurebold}

% font themes: default, professionalfonts, serif, structurebold, structureitalicserif, structuresmallcapsserif

\setbeamerfont{title}{shape=\itshape,family=\rmfamily}
%\setbeamercolor{title}{fg=black!80!black,bg=red!90!white}

\logo{} %\logo{\includegraphics[height=0.5cm]{logo.pdf}}
%% Use \insertlogo to insert the logo at place

\title{FOSSEE}
\subtitle{Free and Open source Software for Science and Engineering
Education}
\author[]{}  %\author[Euclid]{Euclid of Alexandria
\institute{IIT Bombay}
\date[]{} %\date[ISPN ’80]{27th International Symposium of Prime Numbers}


\begin{document}

\begin{frame}
	 \titlepage
\end{frame}

\begin{frame}
\section*{Outline}
\end{frame}

%Uncomment lines below to show table of contents at the beginning
%%\section{Overview}
%%\begin{frame}
%%\tableofcontents
%%\end{frame}

\section{Introduction}
\begin{frame}
\frametitle{FOSS}
\begin{definition}
\alert{FOSS} stands for Free and Open Source Software in Science and Engineering.
\end{definition}
\end{frame}

\subsection{What users can do?}
\begin{frame}
\frametitle{What users can do?}
\begin{example}
\begin{itemize}
\item See and modify the source code.\pause
\item Redistribute and improve the source code.\pause
\item Use the software for any purpose.\pause
\end{itemize}
\end{example}
\end{frame}

\subsection{For whom?}
\begin{frame}
\frametitle{FOSS is advantageous for}
\begin{enumerate}
\item Private Industries \pause
\item Entrepreneurs \pause
\item Defence Establishments \pause
\item Research Organisations \pause
\item Academic Institutions\pause
\item Individual User \pause
\end{enumerate}
\begin{block}{FOSS for everyone}
FOSS is good for anyone who uses Computer to get things done quickly and efficiently. 
\end{block}
\end{frame}


\section{FOSSEE}

\subsection{FOSSEE at IITB}
\begin{frame}
\frametitle{FOSSEE}
\begin{block}{Stands for}
Free and Open source Software for Science and Engineering
Education
\end{block}
\vspace{1cm}
\begin{overlayarea}{\textwidth}{5cm}
\only<1>{Based at IIT Bombay}
\only<2>{Funded by MHRD}
\only<3>{Part of National Mission on Education through
ICT}
\end{overlayarea}
\end{frame}

\subsection{Goals}
\begin{frame}
\frametitle{Goals}
\begin{itemize}
\item Promote \alert{FOSS} packages to minimize use of commercial tools in science and engineering education
\item Documenation for supported \alert{FOSS} packages.
\item Awareness among students and teachers about supported \alert{FOSS} packages. 
\end{itemize}
\end{frame}

\subsection{People}
\begin{frame}
\frametitle{People}
\begin{itemize}
\item Prof. Prabhu Ramachandran \\ 
{\footnotesize Aerospace Engineering} \pause
\item Prof. Mani Bhushan \\
{\footnotesize Chemical Engineering}  \pause
\item Prof. Madhu Belur \\
{\footnotesize Electrical Engineering} \pause
\item Prof. Kannan Moudgalya \\
{\footnotesize Chemical Engineering} \pause
\end{itemize}
\end{frame}

\subsection{Block Diagram}
\begin{frame}
\frametitle{FOSSEE Projects}
\begin{center}
\includegraphics[scale=0.3]{blockdiagram.png}
\end{center}
\end{frame}

\subsection{At IITB}
\begin{frame}
\frametitle{FOSSEE focus at IITB}
\begin{itemize}
\item Python family \pause
   \begin{itemize}
   \item Python
   \item NumPy
   \item SciPy
   \item Sage \pause
   \end{itemize}
\item Scilab family \pause
   \begin{itemize}
   \item Scilab
   \item Xcos \pause
   \end{itemize}
\item Other FOSS actively pursued/used \pause
   \begin{itemize}
   \item GNURadio
   \item KiCA
   \item OpenFOAM
   \item Ngspice
   \item \LaTeX % \Latex \Latex \pause
   \end{itemize}
\end{itemize}
\end{frame}

\section{FOSSEE Activities}

\subsection{Thrust Areas}
\begin{frame}
\frametitle{Thrust Activities}
\begin{itemize}
\item SDES Course \pause
\item Workshops  \pause
\item Textbook Companion Project  \pause
\item Spoken Tutorials \pause
\end{itemize}
\end{frame}

\subsection{SDES}
\begin{frame}
\begin{block}{SDES}
Software Development Techniques for Scientists and Engineers
\end{block}
\begin{itemize}
\item Equips a student with various FOSS tools for curricular purposes.
\item For students of BE/BTech and ME/MTech programmes
\item Currently in curriculum : IIT Bombay, Varanasi and Chennai
\item Coordinators' workshop was held in September 2011.
\end{itemize}
\end{frame}

\begin{frame}
\frametitle{Co-ordinator's Workshop}
\begin{itemize}
\item Introduce the faculty of various engineering colleges to advantages of FOSS. \pause
\item Motivate them to include SDES course in their curriculum. \pause
\item Conducted in 2 parts \pause
\end{itemize}
\begin{enumerate}
\item At IIT
\item Through AVIEW \pause
\end{enumerate}
\begin{block}
From all over India, over \alert{1000} Teacher's will be involved.
\end{block}
\end{frame}

\subsection{Workshops}
\begin{frame}
\frametitle{Workshops all over India}
%\includegraphics[scale=1]{workshop.png}
\end{frame}

\subsection{Spoken Tutorials}
\begin{frame}
\frametitle{Spoken Tutorials}
\begin{columns}
\column{.5\textwidth}
\begin{block}{On}
\begin{itemize}
\item Scilab
\item Python - Basic and Advanced
\item \LaTeX
\item Version Control
\item Linux utilities
\end{itemize}
\end{block}
\column{.5\textwidth}
\begin{block}{For}
Self Learning and Teaching purposes.
Over \alert{50} tutorials already completed.
\end{block}
\pause
\end{columns}
\begin{block}{Website}
http://www.spoken-tutorial.org
\end{block}
\end{frame}


\subsection{TextBook Companion Project}

\begin{frame}
\frametitle{TextBook Companion Project}
\begin{description}
Create a repository of reference material in the form of solved problems for Scientific Computing with Open Source tools.
\end{description}
\end{frame}


\section*{}
\frame{
    \begin{center}
        \huge
        Thank you\\ \pause
    \end{center}
}

\end{document} 